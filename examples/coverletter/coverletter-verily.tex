%!TEX TS-program = xelatex
%!TEX encoding = UTF-8 Unicode
% Awesome CV LaTeX Template for Cover Letter
%
% This template has been downloaded from:
% https://github.com/posquit0/Awesome-CV
%
% Authors:
% Claud D. Park <posquit0.bj@gmail.com>
% Lars Richter <mail@ayeks.de>
%
% Template license:
% CC BY-SA 4.0 (https://creativecommons.org/licenses/by-sa/4.0/)
%


%-------------------------------------------------------------------------------
% CONFIGURATIONS
%-------------------------------------------------------------------------------
% A4 paper size by default, use 'letterpaper' for US letter
\documentclass[9pt, a4paper]{maedbh-cv}

% Configure page margins with geometry
\geometry{left=1cm, top=.5cm, right=1cm, bottom=0.8cm, footskip=0cm}

% Specify the location of the included fonts
\fontdir[fonts/]

% Color for highlights
% Awesome Colors: awesome-emerald, awesome-skyblue, awesome-red, awesome-pink, awesome-orange
%                 awesome-nephritis, awesome-concrete, awesome-darknight
\colorlet{awesome}{awesome-skyblue}
% Uncomment if you would like to specify your own color
% \definecolor{awesome}{HTML}{CA63A8}

% Colors for text
% Uncomment if you would like to specify your own color
% \definecolor{darktext}{HTML}{414141}
% \definecolor{text}{HTML}{333333}
% \definecolor{graytext}{HTML}{5D5D5D}
% \definecolor{lighttext}{HTML}{999999}

% Set false if you don't want to highlight section with awesome color
\setbool{acvSectionColorHighlight}{false}

% If you would like to change the social information separator from a pipe (|) to something else
\renewcommand{\acvHeaderSocialSep}{\quad\textbar\quad}

%-------------------------------------------------------------------------------
%	PERSONAL INFORMATION
%	Comment any of the lines below if they are not required
%-------------------------------------------------------------------------------
% Available options: circle|rectangle,edge/noedge,left/right
\photo[noedge,left]{profile.png}

\name{Maedbh}{King}
\position{{\enskip\cdotp\enskip}Cognitive and Computational Neuroscientist{\enskip\cdotp\enskip}UC Berkeley{\enskip\cdotp\enskip}}
\address{Oakland, CA, USA}

\mobile{(510) 570-5306}
\email{maedbhking@gmail.com}
\homepage{www.maedbhking.com}
\github{maedbhk}
%\linkedin{maedbhking}
\googlescholar{YS9zF8gAAAAJ&hl}{Maedbh King}
% \gitlab{gitlab-id}
% \stackoverflow{SO-id}{SO-name}
% \twitter{@twit}
% \skype{skype-id}
% \reddit{reddit-id}
% \medium{madium-id}
% \googlescholar{googlescholar-id}{name-to-display}
%% \firstname and \lastname will be used
% \googlescholar{googlescholar-id}{}
% \extrainfo{extra informations}

%-------------------------------------------------------------------------------
%	LETTER INFORMATION
%	All of the below lines must be filled out
%-------------------------------------------------------------------------------
% The company being applied to
\recipient
  {Verily Life Sciences}
  {269 E Grand Ave.,\\South San Francisco, CA 94080}
% The date on the letter, default is the date of compilation
\letterdate{\today}
% The title of the letter
\lettertitle{Application for Verily Health Platforms Fellowship}
% How the letter is opened
\letteropening{To whom it may concern, }
% How the letter is closed
\letterclosing{Sincerely,}
% Any enclosures with the letter
%\letterenclosure[Attached]{Curriculum Vitae}

%-------------------------------------------------------------------------------
\begin{document}

% Print the header with above personal informations
% Give optional argument to change alignment(C: center, L: left, R: right)
\makecvheader[R]

% Print the footer with 3 arguments(<left>, <center>, <right>)
% Leave any of these blank if they are not needed
\makecvfooter
  {\today}
  {Maedbh King~~~·~~~Cover Letter}
  {}
  
% Print the title with above letter informations
\makelettertitle

%-------------------------------------------------------------------------------
%	LETTER CONTENT
%-------------------------------------------------------------------------------
\begin{cvletter}

\lettersection{About Me}
I'm currently finishing my Ph.D. in cognitive and computational neuroscience at the University of California, Berkeley. I have 6+ years of scientific research experience, and leadership expertise in academia and non-profit work. My research consists of hypothesis- and data-driven experimentation, and in my work, I apply machine learning and statistics to complex neural databases to build models of brain function. I have a strong track record of publishing in high-impact, peer-reviewed scientific journals (>370 citations), presenting at research conferences (>10 proceedings), and making science accessible (>200 downloads of open-access datasets). I also enjoy teaching and mentoring diverse groups of students, from undergraduates at UC Berkeley to incarcerated students at San Quentin State Prison. Furthermore, I believe in leading with equity, and over the past 5 years, I have used data-driven solutions to promote issues concerning transparency and accountability within academia and non-profit organizations. 

\lettersection{Why Verily?}
I was motivated to apply for this fellowship at Verily for a few reasons. First, as a neuroscientist, I am keen to use my scientific and data-driven expertise to work on innovative projects that exist at the intersection of health sciences, technology, and data science. Projects such as Onduo and OneFifteen have enormous potential to transform how individuals are supported in accessing healthcare, and I would like to be involved in the process of translating basic science research into real-world outcomes. Second, the transition from academia to industry can be daunting, but what appeals to me about this fellowship is 1) the opportunity to sample diverse projects as part of cross-functional teams, and 2) the chance to contribute to innovation at Verily through the diverse set of skills I acquired during my Ph.D. I am excited at the prospect of working on fast-paced and collaborative teams, alongside colleagues who are driven, creative, and thoughtful. 

\lettersection{Why Me?}
I have gained a lot of technical expertise during my scientific training, with a particular focus on data science techniques, advanced statistics, and experimental design. However, as every scientist knows, non-technical skills are integral to successful research, from developing efficient communication strategies for managing cross-functional projects, to establishing norms for collaborative decision-making. I believe that I have the skills necessary to work in many different capacities at Verily, from coordinating and developing projects with widespread impact, to overseeing the full life-cycle of a project from concept to final reports. For example, some of my previous work has involved grant writing, editorial duties, conference organization, high-level project planning, patient testing, setting up computational clusters, and leading a team during the COVID-19 pandemic to collect 300+ hours of neuroimaging data. Importantly, some of the most pivotal research I have done during my Ph.D. has been patient-centered, in which I tested individuals with spinocerebellar ataxia on a series of tasks to assess cognitive deficits. This work really drove home the importance of doing basic science research within the broader context of translational clinical solutions, and it is why I am now keen to work on projects that translate scientific knowledge into clinical and public health outcomes. I am very excited about this next step in my career, and I feel prepared to apply all that I have learned, from quantitative methods to communication skills, to high‐impact projects that push the boundaries of technology and biomedical research. 
 

\end{cvletter}

%-------------------------------------------------------------------------------
% Print the signature and enclosures with above letter informations
\makeletterclosing
\end{document}
