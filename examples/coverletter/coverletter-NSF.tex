%!TEX TS-program = xelatex
%!TEX encoding = UTF-8 Unicode
% Awesome CV LaTeX Template for Cover Letter
%
% This template has been downloaded from:
% https://github.com/posquit0/Awesome-CV
%
% Authors:
% Claud D. Park <posquit0.bj@gmail.com>
% Lars Richter <mail@ayeks.de>
%
% Template license:
% CC BY-SA 4.0 (https://creativecommons.org/licenses/by-sa/4.0/)
%


%-------------------------------------------------------------------------------
% CONFIGURATIONS
%-------------------------------------------------------------------------------
% A4 paper size by default, use 'letterpaper' for US letter
\documentclass[9pt, a4paper]{maedbh-cv}

% Configure page margins with geometry
\geometry{left=1cm, top=.5cm, right=1cm, bottom=0.8cm, footskip=0cm}

% Specify the location of the included fonts
\fontdir[fonts/]

% Color for highlights
% Awesome Colors: awesome-emerald, awesome-skyblue, awesome-red, awesome-pink, awesome-orange
%                 awesome-nephritis, awesome-concrete, awesome-darknight
\colorlet{awesome}{awesome-skyblue}
% Uncomment if you would like to specify your own color
% \definecolor{awesome}{HTML}{CA63A8}

% Colors for text
% Uncomment if you would like to specify your own color
% \definecolor{darktext}{HTML}{414141}
% \definecolor{text}{HTML}{333333}
% \definecolor{graytext}{HTML}{5D5D5D}
% \definecolor{lighttext}{HTML}{999999}

% Set false if you don't want to highlight section with awesome color
\setbool{acvSectionColorHighlight}{false}

% If you would like to change the social information separator from a pipe (|) to something else
\renewcommand{\acvHeaderSocialSep}{\quad\textbar\quad}

%-------------------------------------------------------------------------------
%	PERSONAL INFORMATION
%	Comment any of the lines below if they are not required
%-------------------------------------------------------------------------------
% Available options: circle|rectangle,edge/noedge,left/right
\photo[noedge,left]{profile.png}

\name{Maedbh}{King}
\position{{\enskip\cdotp\enskip}PMF Finalist 2022 and PhD Candidate{\enskip\cdotp\enskip}}
\address{67 Glen Avenue, \#202, Oakland, CA, USA}

\mobile{(510) 570-5306}
\email{maedbhking@gmail.com}
\homepage{www.maedbhking.com}
\github{maedbhk}
%\linkedin{maedbhking}
\googlescholar{YS9zF8gAAAAJ&hl}{Maedbh King}
% \gitlab{gitlab-id}
% \stackoverflow{SO-id}{SO-name}
% \twitter{@twit}
% \skype{skype-id}
% \reddit{reddit-id}
% \medium{madium-id}
% \googlescholar{googlescholar-id}{name-to-display}
%% \firstname and \lastname will be used
% \googlescholar{googlescholar-id}{}
% \extrainfo{extra informations}

%-------------------------------------------------------------------------------
%	LETTER INFORMATION
%	All of the below lines must be filled out
%-------------------------------------------------------------------------------
% The company being applied to
\recipient
  {National Science Foundation}
  {2415 Eisenhower Avenue\\Alexandria, Virginia 22314}
% The date on the letter, default is the date of compilation
\letterdate{\today}
% The title of the letter
\lettertitle{Application for Interdisciplinary Science Analyst}
% How the letter is opened
\letteropening{To whom it may concern, }
% How the letter is closed
\letterclosing{Sincerely,}
% Any enclosures with the letter
%\letterenclosure[Attached]{Curriculum Vitae}

%-------------------------------------------------------------------------------
\begin{document}

% Print the header with above personal informations
% Give optional argument to change alignment(C: center, L: left, R: right)
\makecvheader[R]

% Print the footer with 3 arguments(<left>, <center>, <right>)
% Leave any of these blank if they are not needed
\makecvfooter
  {\today}
  {Maedbh King~~~·~~~Cover Letter}
  {}
  
% Print the title with above letter informations
\makelettertitle

%-------------------------------------------------------------------------------
%	LETTER CONTENT
%-------------------------------------------------------------------------------
\begin{cvletter}

\lettersection{About Me}
I'm currently finishing my Ph.D. in psychology (cognitive neuroscience) at the University of California, Berkeley. I have 6+ years of hypothesis- and data-driven research experience in the social sciences. In my work, I apply machine learning and statistics to large and complex databases to build models of brain function. I have a strong track record of publishing in high-impact, peer-reviewed scientific journals (>350 citations), presenting at research conferences (>10 proceedings), and making science accessible (>200 downloads of open-access datasets and code). I am passionate about teaching and mentoring, and I have experience working with diverse groups of students, from undergraduates at UC Berkeley, to incarcerated students at San Quentin State Prison. Furthermore, my research interests dovetail strongly with science and technology policy, and over the past 5 years, I have used data-driven solutions to promote issues concerning equity, transparency, and accountability within academia and non-profit organizations. 

\lettersection{Why NCSES?}
As a social scientist, I am keen to use my scientific training, particularly my expertise in data science and statistics, to work on innovative projects that ensure U.S. competitiveness in science, engineering, and technology. An analytic position perfectly marries my experience in research and science communication with my passion for data wrangling and mathematical reasoning. In addition, the PMF program at NCSES offers a unique opportunity for professional development training, and I am very eager to hone my leadership skills through meaningful work assignments. Most importantly, I am excited at the prospect of working in a highly interdisciplinary position, alongside colleagues who, as well as being innovative thinkers, are creative, thoughtful, and driven. 

\lettersection{Why Me?}
I have gained a lot of technical expertise during my Ph.D. with a particular focus on data science techniques and advanced statistics. However, as every scientist knows, non-technical skills are integral to successful research, from developing efficient communication strategies for managing cross-functional projects, to establishing norms and guidelines for collaborative decision-making. I believe that I can use these skills to coordinate and develop data projects with national impact, overseeing the full life-cycle from concept to final reports. My previous work has  required a high level of adaptability and critical thinking, from grant/paper writing and editorial duties, to high-level project planning and coordination of goals across interdisciplinary teams. I am excited about my future in public service, and I am especially keen to apply the skills I acquired during my PhD to high impact projects that advance research and development in the U.S.
 
\end{cvletter}

%-------------------------------------------------------------------------------
% Print the signature and enclosures with above letter informations
\makeletterclosing
\end{document}
