%!TEX TS-program = xelatex
%!TEX encoding = UTF-8 Unicode
% Awesome CV LaTeX Template for Cover Letter
%
% This template has been downloaded from:
% https://github.com/posquit0/Awesome-CV
%
% Authors:
% Claud D. Park <posquit0.bj@gmail.com>
% Lars Richter <mail@ayeks.de>
%
% Template license:
% CC BY-SA 4.0 (https://creativecommons.org/licenses/by-sa/4.0/)
%


%-------------------------------------------------------------------------------
% CONFIGURATIONS
%-------------------------------------------------------------------------------
% A4 paper size by default, use 'letterpaper' for US letter
\documentclass[11pt, a4paper]{maedbh-cv}

% Configure page margins with geometry
\geometry{left=1.4cm, top=.8cm, right=1.4cm, bottom=1.8cm, footskip=.5cm}

% Specify the location of the included fonts
\fontdir[fonts/]

% Color for highlights
% Awesome Colors: awesome-emerald, awesome-skyblue, awesome-red, awesome-pink, awesome-orange
%                 awesome-nephritis, awesome-concrete, awesome-darknight
\colorlet{awesome}{awesome-skyblue}
% Uncomment if you would like to specify your own color
% \definecolor{awesome}{HTML}{CA63A8}

% Colors for text
% Uncomment if you would like to specify your own color
% \definecolor{darktext}{HTML}{414141}
% \definecolor{text}{HTML}{333333}
% \definecolor{graytext}{HTML}{5D5D5D}
% \definecolor{lighttext}{HTML}{999999}

% Set false if you don't want to highlight section with awesome color
\setbool{acvSectionColorHighlight}{false}

% If you would like to change the social information separator from a pipe (|) to something else
\renewcommand{\acvHeaderSocialSep}{\quad\textbar\quad}

%-------------------------------------------------------------------------------
%	PERSONAL INFORMATION
%	Comment any of the lines below if they are not required
%-------------------------------------------------------------------------------
% Available options: circle|rectangle,edge/noedge,left/right
\photo[noedge,left]{profile.png}

\name{Maedbh}{King}
\position{{\enskip\cdotp\enskip}Ph.D. Candidate{\enskip\cdotp\enskip}}
\address{67 Glen Avenue, \#202, Oakland, CA, USA}

\mobile{(510) 570-5306}
\email{maedbhking@gmail.com}
\homepage{www.maedbhking.com}
\github{maedbhk}
%\linkedin{maedbhking}
\googlescholar{YS9zF8gAAAAJ&hl}{Maedbh King}
% \gitlab{gitlab-id}
% \stackoverflow{SO-id}{SO-name}
% \twitter{@twit}
% \skype{skype-id}
% \reddit{reddit-id}
% \medium{madium-id}
% \googlescholar{googlescholar-id}{name-to-display}
%% \firstname and \lastname will be used
% \googlescholar{googlescholar-id}{}
% \extrainfo{extra informations}

%-------------------------------------------------------------------------------
%	LETTER INFORMATION
%	All of the below lines must be filled out
%-------------------------------------------------------------------------------
% The company being applied to
\recipient
  {School of Information}
  {102 S Hall\\University of California, Berkeley\\ Berkeley, CA 94720}
% The date on the letter, default is the date of compilation
\letterdate{\today}
% The title of the letter
\lettertitle{Application for Graduate Certificate in Applied Data Science}
% How the letter is opened
\letteropening{To whom it may concern, }
% How the letter is closed
\letterclosing{Sincerely,}
% Any enclosures with the letter
%\letterenclosure[Attached]{Curriculum Vitae}


%-------------------------------------------------------------------------------
\begin{document}

% Print the header with above personal informations
% Give optional argument to change alignment(C: center, L: left, R: right)
\makecvheader[R]

% Print the footer with 3 arguments(<left>, <center>, <right>)
% Leave any of these blank if they are not needed
\makecvfooter
  {\today}
  {Maedbh King~~~·~~~Cover Letter}
  {}

% Print the title with above letter informations
\makelettertitle

%-------------------------------------------------------------------------------
%	LETTER CONTENT
%-------------------------------------------------------------------------------
\begin{cvletter}

I'm currently a PhD candidate in cognitive neuroscience at UC Berkeley. I design large-scale neuroimaging studies and use machine learning to build predictive models of cerebellar function. The holy grail of cognitive neuroscience is to understand what the cerebellum is doing in cognition. My research attempts to answer this question, and the more I learn, the more I am fascinated by this mysterious yet widely misunderstood brain structure.

All of the work that I have conducted thus far in my PhD has relied on data science techniques. For example, I have used semi non-negative matrix factorization to build a functional map of the cerebellum, and feature-based regularized regression to estimate and test models of cerebro-cerebellar connectivity. Dimensionality reduction (e.g., PCA) and hierarchical clustering are the bread and butter for a lot of the analyses that I do in analyzing high-dimensional brain data. I could not do my research without the principles and techniques afforded by advances in data science. 

Thank you in advance for considering my application. 

\end{cvletter}


%-------------------------------------------------------------------------------
% Print the signature and enclosures with above letter informations
\makeletterclosing

\end{document}
