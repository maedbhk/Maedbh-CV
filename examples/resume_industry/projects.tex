%-------------------------------------------------------------------------------
%	SECTION TITLE
%-------------------------------------------------------------------------------
\cvsection{Selected Projects}


%-------------------------------------------------------------------------------
%	CONTENT
%-------------------------------------------------------------------------------
\begin{cventries}
    
%---------------------------------------------------------
  \cventry
    {2021} % Date(s)
  	{SUITPy: Open-source package for the visualization of cerebellum imaging data} % Organization
  	{\href{https://github.com/DiedrichsenLab/SUITPy}{\textbf{Github [Link]}}}
  	{}
    {
      \begin{cvitems} % Description(s) of tasks/responsibilities
      	\item {Core developer of \textit{SUITPy}, an open-source \textit{python} toolbox based on a highly popular \textit{MATLAB} toolbox. I implemented \textit{mapping} of brain data to 2D surface space and incorporated \textit{brain atlases} from \textit{open-source} repositories.}
      	%\item {Incorporated \textbf{HTTP} requests to fetch \textbf{brain atlases} from \textbf{open-source} repositories and improved the \textbf{mapping} of brain data to 2D surface space.}
      \end{cvitems}
    }
    
%---------------------------------------------------------
  \cventry
    {2021} % Date(s)
    {Evaluating functional boundaries of the brain using a novel distance coefficient} % Organization
    {\href{https://www.biorxiv.org/content/10.1101/2021.05.11.443151v1.full.pdf}{\textbf{Paper [Link]}}}
    {}
    {
      \begin{cvitems} % Description(s) of tasks/responsibilities
        \item {Co-developed a novel \textit{statistical metric} to evaluate the validity of \textit{brain parcellations}, an advancement on \textit{Homogeneity and Silhouette coefficients}. Evaluated metric on \textit{open-source} brain data from \href{http://www.humanconnectomeproject.org/}{\textit{Human Connectome Project}}.}
       % \item {This method is a big improvement on \textit{Homogeneity and Silhouette coefficients} traditionally used in \textbf{brain parcellation research}.}
      \end{cvitems}
    }
    
 %---------------------------------------------------------
  \cventry
    {2021} % Date(s)
    {Predicting brain activation maps for arbitrary tasks with cognitive encoding models} % Organization
    {\href{https://drive.google.com/file/d/1dRNSsRzGSSF9QJLJv_jK1R7BmmWfOsMt/view}{\textbf{Poster [Link]}}}
    {}
    {
      \begin{cvitems} % Description(s) of tasks/responsibilities
      	\item {Evaluated cognitive \textit{encoding models} on brain data and used \textit{natural language processing} to extract features from a formal \textit{cognitive ontology}.}
        %\item {Used feature-based encoding to find important gene samples in the cerebellum, and \textbf{hierarchical clustering and PCA} (using scikit-learn) to determine \textbf{organizational structure} of genetic gradients in a \textbf{low dimensional} space.}
      \end{cvitems}
    }   
    
    
%---------------------------------------------------------
  \cventry
    {2021} % Date(s)
    {Low dimensional embedding of genetic gradients in the human cerebellum} % Organization
    {\href{https://papers.ssrn.com/sol3/papers.cfm?abstract_id=3797269}{\textbf{Paper [Link]}}}
    {}
    {
      \begin{cvitems} % Description(s) of tasks/responsibilities
      	\item {Investigated \textit{genetic gradients} in the \textit{human cerebellum} using postmortem data from the \href{https://human.brain-map.org/}{\textit{Allen Human Brain Atlas}}. Used feature-based encoding to locate gene samples in the cerebellum, and \textit{hierarchical clustering and PCA} to determine \textit{organizational structure} of genetic gradients}
        %\item {Used feature-based encoding to find important gene samples in the cerebellum, and \textbf{hierarchical clustering and PCA} (using scikit-learn) to determine \textbf{organizational structure} of genetic gradients in a \textbf{low dimensional} space.}
      \end{cvitems}
    }
    
%---------------------------------------------------------
  \cventry
    {2020} % Date(s)
    {Predicting penalty shots using markerless pose estimation} % Organization
    {\href{https://github.com/maedbhk/action_prediction}{\textbf{Github [Link]}}}
    {}
    {
      \begin{cvitems} % Description(s) of tasks/responsibilities
      	\item {Implemented \textit{markerless labeling} of video data (>12 hours of soccer players taking penalty shots) and feature-based encoding to \textit{compare model and human performance} in predicting penalty outcomes.}
        %\item {Used feature-based encoding to find important gene samples in the cerebellum, and \textbf{hierarchical clustering and PCA} (using scikit-learn) to determine \textbf{organizational structure} of genetic gradients in a \textbf{low dimensional} space.}
      \end{cvitems}
    }   
    
%---------------------------------------------------------
  \cventry
    {2020} % Date(s)
    {Predicting COVID-19 mortality rates across the U.S. using mobility and census data} % Organization
    {\href{https://drive.google.com/file/d/1l9TLGLmstkJsvOJQPxHe_f35b4fEkDU4/view}{\textbf{Report [Link]}}}
    {}
    {
      \begin{cvitems} % Description(s) of tasks/responsibilities
      	\item {Implemented \textit{elastic net} regularization using \textit{economic} and \textit{mobility} features to \textit{predict} \textit{COVID-19} deaths across the U.S. in 2020 using data from the 2019 \textit{U.S. Census} and \textit{Google Maps} mobility reports.}
      	%\item {Model \textbf{features} were extracted using \textbf{dimensionality reduction} and \textbf{elastic net} regularization and \textbf{ridge regression} was used to \textbf{train and test models}.}
      \end{cvitems}
    }
    
%---------------------------------------------------------
\end{cventries}
