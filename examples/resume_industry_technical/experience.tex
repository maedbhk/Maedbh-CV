%-------------------------------------------------------------------------------
%	SECTION TITLE
%-------------------------------------------------------------------------------
\cvsection{Experience}


%-------------------------------------------------------------------------------
%	CONTENT
%-------------------------------------------------------------------------------
\begin{cventries}

%---------------------------------------------------------
  \cventry
    {Using machine learning to identify risk predictors of neuropsychiatric disorders}
    {ICoN Postdoctoral Fellow at MIT} % Job title
    {\href{https://maedbhk.github.io/MIT-Projects/assets/images/ICoN_profile.png}{\textbf{Fellow Profile [link]}}}
    {2022-} % Organization
    {
      \begin{cvitems} % Description(s) of tasks/responsibilities
		\item {Managed a team of 4 to implement a transdiagnostic approach to determine how brain differences classify neurodevelopmental disorders. Trained an artificial neural network on a sample of 4,000 participants to capture non-linear relationships in cognitive profiles and cortical morphology data.}
		\item {Collaborated with child psychiatrists to identify risk predictors for self-harm and suicide attempt in an adolescent population. Trained natural language processing (NLP) models to decode unstructured electronic health records and predictive modeling to identify critical features from structured clinical profiles. Results will be incorporated into medical framework to inform clinical outcomes.}
		\end{cvitems}
    }


%---------------------------------------------------------
  \cventry
    {Ph.D. Thesis: Mapping cerebro-cerebellar networks of the human brain during learning}
    {PhD Researcher at UC Berkeley} % Job title
    {\href{http://ivrylab.berkeley.edu/uploads/4/1/1/5/41152143/functional_boundaries_in_the_human_cerebellum.pdf}{\textbf{Paper [link]}}}
    {2017-2022} % Organization
    {
      \begin{cvitems} % Description(s) of tasks/responsibilities
            	\item {Created a novel map of the human cerebellum by applying matrix factorization to high-dimensional neural data {\href{http://www.diedrichsenlab.org/imaging/mdtb.htm}		{[\textit{link}}]}.}
        \item {Developed machine learning pipelines to predict cognitive function in the human cerebellum during learning, tested patients with spinocerebellar ataxia on a series of cognitive tasks to assess cerebellar deficits, analyzed post-mortem  human brain data to create a transcriptomic map of the human cerebellum, and led a team of 5 to collect 300 experimental hours of functional magnetic resonance imaging (fMRI) data (during COVID-19 pandemic).}
        \item {Co-wrote an R35 grant that received 5-year funding from the NIH. Managed an institutional review board (IRB) protocol for fMRI experiments.}
        \item {Created a widely adopted mentorship agreement for research assistants to ensure transparency and accountability in mentoring practices. Co-led a journal club for undergraduate research assistants, instructing them on the scientific method, data analysis, and statistics.}
      \end{cvitems}
    }
    
%---------------------------------------------------------
\end{cventries}
