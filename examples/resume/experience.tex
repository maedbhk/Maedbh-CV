%-------------------------------------------------------------------------------
%	SECTION TITLE
%-------------------------------------------------------------------------------
\cvsection{Experience}


%-------------------------------------------------------------------------------
%	CONTENT
%-------------------------------------------------------------------------------
\begin{cventries}

%---------------------------------------------------------
  \cventry
    {University of California, Berkeley} % Organization
    {Thesis: Mapping cerebro-cerebellar networks of the human brain during learning}
    {\href{https://github.com/maedbhk/cerebellum_learning_connect}{\textbf{Github [Link]}}}
    {Graduate Student Researcher (2017-)} % Job title
    {
      \begin{cvitems} % Description(s) of tasks/responsibilities
        \item {The aim of my thesis is to use \textit{machine learning} to predict cognitive function across learning in the human cerebellum using \textit{cortical features}.} 
        \item {\textit{Led a team of 9} (3 Ph.D. students, 5 research assistants, 1 postbac student) to design and collect \textit{300} experimental hours of \textit{fMRI} and \textit{eye-tracking} data.}
        \item {Developed \textit{encoding models} to build an optimal model of \textit{cerebro-cerebellar connectivity}, features were \textit{extracted} by parcellating the human cerebral cortex and \textit{feature selection} was performed with \textit{L1 regularization}.}
        \item {Used \textit{dimensionality reduction} (PCA, ICA), \textit{clustering}, \textit{regression}, \textit{permutation tests} and other machine learning techniques to analyze \textit{behavioral} and \textit{eye-tracking} data to predict human \textit{learning} performance on \textit{movie-based} action prediction tasks.}
        \item {\textit{Created} a widely adopted \textit{mentorship agreement} for research assistants to ensure \textit{transparency and accountability} in mentoring practices. Co-led a \textit{journal club} for undergraduate research assistants, \textit{instructing} them on the \textit{scientific method}.}
      \end{cvitems}
    }
    
%---------------------------------------------------------
  \cventry
    {Western University} % Organization
    {Thesis: Understanding the functional organization of the human cerebellum}
    {\href{http://ivrylab.berkeley.edu/uploads/4/1/1/5/41152143/functional_boundaries_in_the_human_cerebellum.pdf}{\textbf{Paper [Link]}}}
    {Graduate Student Researcher (2015-2017)} % Job title
    {
      \begin{cvitems} % Description(s) of tasks/responsibilities
        \item {My thesis used \textit{machine learning} to map cognitive sub-domains of the human cerebellum.}
      	\item {\textit{Led a team of 2} (1 research assistant and one post-doctoral fellow) to design and collect a 26-task \textit{fMRI} experiment and used \textit{semi non-negative matrix factorization} to generate a \textit{novel functional map} of the human cerebellum.}
      	\item {\textit{Initiated} a \textit{collaboration} with scientists from Stanford University to use \textit{natural language processing and regularized regression} to assign cognitive labels ( {\href{https://cognitiveatlas.org/}{\textit{cognitiveatlas.org}}}) to the human cerebellum.}
      	\item {My datasets are publicly available on {\href{https://openneuro.org/datasets/ds002105/versions/1.1.0}{\textit{openneuro.org}}}, and have been downloaded by \textit{hundreds of researchers}.}
      \end{cvitems}
    }

%---------------------------------------------------------
\end{cventries}
