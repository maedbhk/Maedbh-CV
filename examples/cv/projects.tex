%-------------------------------------------------------------------------------
%	SECTION TITLE
%-------------------------------------------------------------------------------
\cvsection{Experience}


%-------------------------------------------------------------------------------
%	CONTENT
%-------------------------------------------------------------------------------
\begin{cventries}

%---------------------------------------------------------
  \cventry
    {University of California, Berkeley} % Organization
    {Cerebellum Learning Project} % Job title
    {Berkeley, California} % Location
    {2019 - } % Date(s)
    {
      \begin{cvitems} % Description(s) of tasks/responsibilities
        \item {Used visualization tools: seaborn, matplotlib, pyplot to prepare figures for paper presentation.}
        \item {Used high performance computing cluster (savio) to run preprocessing and connectivity scripts.}
        \item {Wrote custom python scripts incorporating machine-learning tools: feature-based encoding, dimensionality reductions, regularized regression, to build connectivity models predicting cerebellar activation across learning.}
        \item {Wrote custom python scripts to preprocess fMRI data, utilizing fmriprep, freesurfer, nipype, scikit-learn etc.}
        \item {Recruited three undergraduate research assistants to collect behavioral and eyetracking data from 25 participants, totaling 100 hours, including behavioral piloting. In addition, I spent 300 hours collecting fMRI data.}
        \item {Used a python library (psychopy) to design a multi-task, multi-session fMRI experiment, the goal of which was to build models that could use activation from the cerebral cortex to predict activation in the cerebellum across learning in the human brain.}
      \end{cvitems}
    }
    
%---------------------------------------------------------
  \cventry
    {University of California, Berkeley} % Organization
    {Cerebellum Transcriptomics Project} % Job title
    {Berkeley, California} % Location
    {2019 - 2021} % Date(s)
    {
      \begin{cvitems} % Description(s) of tasks/responsibilities
      	\item {Presented findings at domestic and international conferences and in a peer-reviewed paper.}
        \item {Used machine-learning tools: feature-based encoding, unsupervised learning including hierarchical clustering and principal component analysis, to cluster gene samples across the cerebellar cortex.}
        \item {Wrote custom python scripts utilizing python libraries (e.g., abagen) to preprocess transcriptomic data.}
        \item {Analyzed transcriptomic and MRI data from 6 postmortem human cerebella (data provided by Allen Human Brain Atlas). The goal of the project was to investigate genetic gradients in the human cerebellum.}
      \end{cvitems}
    }

%---------------------------------------------------------
  \cventry
    {University of California, Berkeley} % Organization
    {Cerebellum Connectivity Project} % Job title
    {Berkeley, California} % Location
    {2017 - 2021} % Date(s)
    {
      \begin{cvitems} % Description(s) of tasks/responsibilities
        \item {Used visualization tools: seaborn, matplotlib, pyplot to prepare figures for paper presentation.}
        \item {Used high performance computing cluster (savio) to run preprocessing and connectivity scripts.}
        \item {Wrote custom python scripts incorporating machine-learning tools: feature-based encoding, dimensionality reductions, regularized regression, to build connectivity models predicting cerebellar activation.}
        \item {This project was the result of an international collaboration with researchers in Canada, the goal of which was to map connectivity patterns between the cerebral cortex and the cerebellum in the human brain.}
      \end{cvitems}
    }
    
%---------------------------------------------------------
  \cventry
    {University of California, Berkeley} % Organization
    {Cerebellum Language Project} % Job title
    {Berkeley, California} % Location
    {2019 - 2021} % Date(s)
    {
      \begin{cvitems} % Description(s) of tasks/responsibilities
        \item {Supervised the student in preparing and presenting her findings at a department colloquium. I also supervised the student in writing her honors thesis on the project.}
        \item {Recruited and mentored an undergraduate student to analyze the patient data using custom written python scripts.}
        \item {Tested patients with spinocerebellar ataxia to investigate whether the cerebellum plays an important role in language production.}
      \end{cvitems}
    }
    
%---------------------------------------------------------
  \cventry
  	{University of California, Berkeley} % Organization
    {Cerebellum Mapping Project} % Job title
    {Berkeley, California} % Location
    {2017 - 2019} % Date(s)
    {
      \begin{cvitems} % Description(s) of tasks/responsibilities
      	\item {Presented findings at domestic and international conferences and in a peer-reviewed paper.}
        \item {The evaluation was done using open-source datasets, Human Connectome Project and Multi-Domain Task Battery.}
        \item {The goal of this project was to evaluate functional boundaries of the human cerebellum using a novel statistical metric known as the boundary-controlled distance coefficient.}
      \end{cvitems}
    }
    
%---------------------------------------------------------
  \cventry
  	{Western University} % Organization
    {Multi-Domain Task Battery Project} % Job title
    {London, Ontario} % Location
    {2015 - 2017} % Date(s)
    {
      \begin{cvitems} % Description(s) of tasks/responsibilities
        \item {The rich dataset that I generated has been made publicly available on openneuro.org and has been downloaded by hundreds of researchers.}
        \item {Used semi non-negative matrix factorization to generate a functional map of the human cerebellum.}
        \item {Collected fMRI and eyetracking data from 25 participants, totaling 400 hours, including behavioral piloting.}
        \item {Designed a multi-session, multi-task experiment with the goal of mapping the human cerebellum.}
      \end{cvitems}
    }

%---------------------------------------------------------
\end{cventries}
