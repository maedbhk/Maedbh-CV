%-------------------------------------------------------------------------------
%	SECTION TITLE
%-------------------------------------------------------------------------------
\cvsection{Experience}


%-------------------------------------------------------------------------------
%	CONTENT
%-------------------------------------------------------------------------------
\begin{cventries}

%---------------------------------------------------------
  \cventry
  	{Mapping networks of the human brain during learning. Publications in \textit{Nature Neuroscience}, \textit{Neuron}, \textit{Brain}}
    {Ph.D. Graduate Student Researcher (2017-2022)} % Job title
    {\href{https://github.com/maedbhk/cerebellum_learning_connect}{\textbf{Github [Link]}}}
    {University of California, Berkeley} % Organization
    {
      \begin{cvitems} % Description(s) of tasks/responsibilities
        \item {Developed machine learning pipelines to predict cognitive function in the human cerebellum during learning, tested patients with spinocerebellar ataxia on a series of cognitive tasks to assess cerebellar deficits, and analyzed post-mortem brain data to create a transcriptomic map of the cerebellum.}
        \item {Prioritized taking classes in advanced statistics and computer science to analyze high-dimensional neural data.} 
        \item {Led a team of 5 to design and collect 300 experimental hours of functional magnetic resonance imaging (fMRI) data (during COVID-19 pandemic).}
		\item {Reviewed and analyzed 30 years of research to propose new theory of cerebellar function. Co-authored subsequent paper \href{https://www.sciencedirect.com/science/article/pii/S0896627319303782}{\textit{[link]}} (>70 citations).}
        \item {Co-wrote an R35 grant that received 5-year funding from the NIH. Managed an institutional review board (IRB) protocol for fMRI experiments.}
        \item {Created a widely adopted mentorship agreement for research assistants to ensure transparency and accountability in mentoring practices.}
        \item {Co-led a journal club for undergraduate research assistants, instructing them on the scientific method, data analysis, and statistics.}
        \item {Mentored two undergraduate students in designing and writing up honors thesis projects. One student was awarded a \href{https://psychology.berkeley.edu/sites/default/files/undergraduate-program/swanaward_application_2019-2020.pdf}{\textit{Swan prize [link]}} for their work, and presented at a national conference. Prioritized recruitment of students from marginalized demographics.}
      \end{cvitems}
    }
    
%---------------------------------------------------------
  \cventry
    {Developing novel brain maps of the human cerebellum. Publications in \textit{Brain}, \textit{NeuroImage}, \textit{Frontiers}}
    {M.Sc. Graduate Student Researcher (2015-2017)} % Job title
    {\href{http://ivrylab.berkeley.edu/uploads/4/1/1/5/41152143/functional_boundaries_in_the_human_cerebellum.pdf}{\textbf{Paper [Link]}}}
    {Western University} % Organization
    {
      \begin{cvitems} % Description(s) of tasks/responsibilities
        \item {Created a novel and highly downloaded {\href{http://www.diedrichsenlab.org/imaging/mdtb.htm}{\textit{ map [link]}}} of the human cerebellum using machine learning and advanced statistics.}
      	\item {Led a team of 2 to design and collect a 26-task, 6-session fMRI and behavioral experiment, totaling 250 hours of data collection {\href{http://www.diedrichsenlab.org/imaging/AtlasViewer/viewer.html}{\textit{[link]}}}.}
      	\item {Initiated a collaboration with scientists from Stanford University to use natural language processing to assign cognitive labels to the human cerebellum {\href{https://cognitiveatlas.org/}{\textit{[link]}}}. Developed programming pipeline for other researchers to replicate my novel approach, and wrote supporting documentation.}
      	\item {Invested in open-source science. My data, which are publicly available, have been downloaded >200 times {\href{https://openneuro.org/datasets/ds002105/versions/1.1.0}{\textit{[link]}}}.}
      \end{cvitems}
    }

%---------------------------------------------------------
\end{cventries}
