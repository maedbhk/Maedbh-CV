%-------------------------------------------------------------------------------
%	SECTION TITLE
%-------------------------------------------------------------------------------
\cvsection{Selected Projects}


%-------------------------------------------------------------------------------
%	CONTENT
%-------------------------------------------------------------------------------
\begin{cventries}
    
%---------------------------------------------------------
  \cventry
    {2021} % Date(s)
  	{SUITPy: Analysis and visualization of cerebellum imaging data} % Organization
  	{\href{https://github.com/DiedrichsenLab/SUITPy}{\textbf{Github [Link]}}}
  	{}
    {
      \begin{cvitems} % Description(s) of tasks/responsibilities
      	\item {A core developer of \textbf{SUITPy}, a \textbf{python} toolbox based on a highly popular MATLAB toolbox, to analyze and visualize cerebellum data.}
      	\item {Incorporated \textbf{HTTP} requests to fetch \textbf{brain atlases} from \textbf{open-source} repositories and improved the \textbf{mapping} of brain data to 2D surface space.}
      \end{cvitems}
    }
    
%---------------------------------------------------------
  \cventry
    {2021} % Date(s)
    {Evaluating functional boundaries of the brain using a boundary-controlled distance coefficient} % Organization
    {\href{https://www.biorxiv.org/content/10.1101/2021.05.11.443151v1.full.pdf}{\textbf{Paper [Link]}}}
    {}
    {
      \begin{cvitems} % Description(s) of tasks/responsibilities
        \item {Co-developed a novel \textbf{statistical metric} to evaluate the strength of functional boundaries in the \textbf{human brain}, evaluating open-source brain data from \href{http://www.humanconnectomeproject.org/}{\textbf{Human Connectome Project}}}
        \item {This method is a big improvement on \textbf{Homogeneity and Silhouette coefficients} traditionally used in \textbf{brain parcellation research}.}
      \end{cvitems}
    }
    
%---------------------------------------------------------
  \cventry
    {2020} % Date(s)
    {Low dimensional embedding of genetic gradients in the human cerebellum} % Organization
    {\href{https://papers.ssrn.com/sol3/papers.cfm?abstract_id=3797269}{\textbf{Paper [Link]}}}
    {}
    {
      \begin{cvitems} % Description(s) of tasks/responsibilities
      	\item {The goal of the project was to investigate \textbf{genetic gradients} in the \textbf{human cerebellum} using \textbf{open-source} postmortem data from the \href{https://human.brain-map.org/}{\textbf{Allen Human Brain Atlas}}.}
        \item {Used feature-based encoding to find important gene samples in the cerebellum, and \textbf{hierarchical clustering and PCA} (using scikit-learn) to determine \textbf{organizational structure} of genetic gradients in a \textbf{low dimensional} space.}
      \end{cvitems}
    }
    
%---------------------------------------------------------
  \cventry
    {2020} % Date(s)
    {Predicting COVID-19 mortality rates across the U.S. using mobility and census data} % Organization
    {\href{https://drive.google.com/file/d/1l9TLGLmstkJsvOJQPxHe_f35b4fEkDU4/view}{\textbf{Report [Link]}}}
    {}
    {
      \begin{cvitems} % Description(s) of tasks/responsibilities
      	\item {Used \textbf{economic} and \textbf{mobility} factors to \textbf{predict} \textbf{COVID-19} deaths across the U.S. in 2020 using data from the 2019 \textbf{U.S. Census} and \textbf{Google Maps} mobility reports.}
      	\item {Model \textbf{features} were extracted using \textbf{dimensionality reduction} and \textbf{elastic net} regularization and \textbf{ridge regression} was used to \textbf{train and test models}.}
      \end{cvitems}
    }
    
%---------------------------------------------------------
\end{cventries}
