%-------------------------------------------------------------------------------
%	SECTION TITLE
%-------------------------------------------------------------------------------
\cvsection{Experience}


%-------------------------------------------------------------------------------
%	CONTENT
%-------------------------------------------------------------------------------
\begin{cventries}

%---------------------------------------------------------
  \cventry
    {University of California, Berkeley} % Organization
    {Thesis: Mapping cerebro-cerebellar networks of the human brain during learning}
    {\href{https://github.com/maedbhk/cerebellum_learning_connect}{\textbf{Github [Link]}}}
    {Graduate Student Researcher (2017-)} % Job title
    {
      \begin{cvitems} % Description(s) of tasks/responsibilities
        \item {The aim of my thesis is to use \textbf{machine learning} to predict cognitive function across learning in the human cerebellum using \textbf{cortical features}.} 
        \item {\textbf{Led a team of 9} (3 Ph.D. students, 5 research assistants, 1 postbac student) to design and collect \textbf{300} experimental hours of \textbf{fMRI} and \textbf{eye-tracking} data.}
        \item {Developed \textbf{encoding models} to build an optimal model of \textbf{cerebro-cerebellar connectivity}, features were \textbf{extracted} by parcellating the human cerebral cortex and \textbf{feature selection} was performed using \textbf{supervised learning} (L1 regularization).}
        \item {Used \textbf{dimensionality reduction} (PCA, ICA, semi-nonnegative matrix factorization), \textbf{clustering}, \textbf{regression}, \textbf{permutation tests} and other machine learning techniques to analyze \textbf{behavioral} and \textbf{eye-tracking} data to predict human \textbf{learning} performance on \textbf{movie-based} action prediction tasks.}
      \end{cvitems}
    }
    
%---------------------------------------------------------
  \cventry
    {Western University} % Organization
    {Thesis: Understanding the functional organization of the human cerebellum}
    {\href{http://ivrylab.berkeley.edu/uploads/4/1/1/5/41152143/functional_boundaries_in_the_human_cerebellum.pdf}{\textbf{Paper [Link]}}}
    {Graduate Student Researcher (2015-2017)} % Job title
    {
      \begin{cvitems} % Description(s) of tasks/responsibilities
        \item {My thesis used \textbf{machine learning} to map cognitive sub-domains of the human cerebellum.}
      	\item {I designed and collected a 26-task fMRI experiment and used \textbf{semi non-negative matrix factorization} to generate a \textbf{novel functional map} of the human cerebellum.}
      	\item {I used \textbf{feature-based encoding models} and \textbf{natural language processing} to assign cognitive labels (sourced from {\href{https://cognitiveatlas.org/}{\textbf{cognitiveatlas.org}}}) to functional domains of the cerebellum.}
      	\item {The rich dataset that I generated has been made publicly available on {\href{https://openneuro.org/datasets/ds002105/versions/1.1.0}{\textbf{openneuro.org}}} and has been downloaded by hundreds of researchers.}
      \end{cvitems}
    }

%---------------------------------------------------------
\end{cventries}
